
\documentclass{article}
\usepackage{verbatim}

% https://tex.stackexchange.com/questions/6073
% https://tex.stackexchange.com/questions/195521
% I'm just using this package to add the "max width" optional key to
% includegraphics. Otherwise, it just wraps graphicx as far as I'm
% concerned.
\usepackage[export]{adjustbox}

\usepackage{fancyhdr}

\usepackage{figureSeries}

\usepackage{geometry}
% Comment after finding appropriate frame boundaries.
%\usepackage{showframe}
% From: https://tex.stackexchange.com/questions/39383/determine-text-width
% Required for printing \textwidth, etc, w/ particular units below.
%\usepackage{layouts}

% TODO what was this for?
\newgeometry{vmargin={15mm}, hmargin={12mm,17mm}}

% not sure this is working as intended... section seems displayed twice, once again
% before first figure in section
%
% If you don't want the "Figure: " prefix
\captionsetup[figure]{labelformat=empty}

% TODO what was this for?
\captionsetup[sub]{labelformat=empty}

% Some stack overflow comment said that although the \verb functionality is in LaTeX by
% default, the implementations in this package are "better", and override the defaults.
\usepackage{verbatimbox}

\usepackage{hyperref}
\hypersetup{
    colorlinks=true,
    linkcolor=blue,
    filecolor=magenta,
    urlcolor=cyan,
}
\urlstyle{same}

% TODO can plots be in any subdirectory? what if i have multiple directories i wanna
% pull from (tested to work as long as paths are relative to fig_dir)?
\graphicspath{{\VAR{fig_dir}/}}


\BLOCK{ if header }

% For headers
\pagestyle{fancy}
% Clear existing header/footer entries
\fancyhf{}

\BLOCK{ endif }


\begin{document}

\BLOCK{ if header }
\begin{verbbox}\VAR{header}\end{verbbox}
\fancyhead[C]{\theverbbox}
\BLOCK{ endif }

% TODO try to replace section[0/1] w/ named variables, if jinja syntax supports
\BLOCK{ for section in sections }
% TODO what is stored in section[0] and section[1]? (section name and list of plots i
% think)
% TODO replace w/ just \section (and remove section above?)
%%\subsection{\VAR{section[0]}}
%\section{\VAR{section[0]}}

% figureSeriesFloat seems to produces slightly more consistent spacing (from the top of
% the page to the first figure, for a figureSeries spanning multiple pages), than
% figureSeriesHere does, but still unsure how to get it totally consistent.
\figureSeriesFloat{}
%\figureSeriesHere{}
% TODO maybe just make the section names take up less space (/ add to a variable part of
% header maybe?). make display conditional on some flag in python fn compiling this?
%\figureSeriesHere{\VAR{section[0]}}

\BLOCK{ for fig_path in section[1] }

\BLOCK{ if filename_captions }
\begin{verbbox}\VAR{fig_path}\end{verbbox}
\BLOCK{ endif }

% Full `max height=\textheight` didn't prevent a blank page if first figure was too
% large, but not sure the maximum fraction of \textheight that would also acheive this.
%
% Lowested tested fraction that also resulted in a blank page before first figure: 0.91
\figureSeriesRow{
\figureSeriesElement{\BLOCK{ if filename_captions }\theverbbox\BLOCK{ endif
}}{\includegraphics[max width=\textwidth,max height=0.9\textheight,keepaspectratio]{\VAR{fig_path}}}
}

%# TODO is this a valid test in this context?
%\BLOCK{ if fig_path in pagebreak_after }
%\pagebreak
%\BLOCK{ endif }

\BLOCK{ endfor }
\pagebreak

\BLOCK{ endfor }

\end{document}
