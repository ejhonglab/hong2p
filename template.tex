
\documentclass{article}
\usepackage{graphicx}
\usepackage{figureSeries}

\usepackage{geometry}
% Comment after finding appropriate frame boundaries.
%\usepackage{showframe}
% From: https://tex.stackexchange.com/questions/39383/determine-text-width
% Required for printing \textwidth, etc, w/ particular units below.
%\usepackage{layouts}

\newgeometry{vmargin={15mm}, hmargin={12mm,17mm}}

% If you don't want the "Figure: " prefix
%\captionsetup[figure]{labelformat=empty}
\captionsetup[sub]{labelformat=empty}

\BLOCK{ if filename_captions }
    % Some stack overflow comment said that although the \verb functionality
    % is in LaTeX by default, the implementations in this package are "better"
    % , and override the defaults.
    \usepackage{verbatim}
    \usepackage{cprotect}
\BLOCK{ endif }

\graphicspath{{\VAR{pdfdir}/}}

\begin{document}

\title{Complex-mixture experiments analysis}
\author{Tom O'Connell}
\date{\today}
\maketitle

% Using the layouts package referenced above:
%\printinunitsof{in}\prntlen{\textwidth} % was 7.35971in
%\printinunitsof{in}\prntlen{\textheight} % was 9.82082in

\pagebreak

% TODO some way to reduce vspace between rows a bit? couldn't tell from
% figureSeries docs...
% w/ default vspace, 0.25\textheight seems about most I can do to fit 3 rows
% (0.3 was too much)

% More than ~0.49 times \textwidth seems to breakup the rows.
% Goal is to get 2 figures included per row, to have control and kiwi
% experiments (within each fly) side-by-side.

\BLOCK{ for section in sections }
%#\section{\VAR{section[0]}}

\figureSeriesHere{\VAR{section[0]}}

\BLOCK{ for row_figpdfs in section[1]|batch(2) }
\figureSeriesRow{
\BLOCK{ for figpdf in row_figpdfs }
\figureSeriesElement{}{\includegraphics[width=0.49\textwidth,height=0.25\textheight,keepaspectratio]{\VAR{figpdf}}}
\BLOCK{ endfor }
}
\BLOCK{ endfor }
%# Was previously in a single figure definition in loop over figpdfs:
%# % https://tex.stackexchange.com/questions/8810
%# \BLOCK{ if filename_captions }
%# \cprotect\caption{\verb=\VAR{figpdf}=}
%# \BLOCK{ endif }
%# See: https://tex.stackexchange.com/questions/122786
%# if trying to adapt to subfloat case.

\pagebreak

\BLOCK{ endfor }

\end{document}
