
\documentclass{article}
\usepackage{verbatim}
\usepackage{graphicx}
\usepackage{figureSeries}

\usepackage{geometry}
% Comment after finding appropriate frame boundaries.
%\usepackage{showframe}
% From: https://tex.stackexchange.com/questions/39383/determine-text-width
% Required for printing \textwidth, etc, w/ particular units below.
%\usepackage{layouts}

\newgeometry{vmargin={15mm}, hmargin={12mm,17mm}}

% If you don't want the "Figure: " prefix
%\captionsetup[figure]{labelformat=empty}
\captionsetup[sub]{labelformat=empty}

\BLOCK{ if filename_captions }
    % Some stack overflow comment said that although the \verb functionality
    % is in LaTeX by default, the implementations in this package are "better"
    % , and override the defaults.
    \usepackage{cprotect}
\BLOCK{ endif }

\usepackage{hyperref}
\hypersetup{
    colorlinks=true,
    linkcolor=blue,
    filecolor=magenta,      
    urlcolor=cyan,
}
\urlstyle{same}

\graphicspath{{\VAR{pdfdir}/}}

\begin{document}

\title{Complex-mixture experiments analysis}
\author{Tom O'Connell}
\date{\today}
\maketitle

% Using the layouts package referenced above:
%\printinunitsof{in}\prntlen{\textwidth} % was 7.35971in
%\printinunitsof{in}\prntlen{\textheight} % was 9.82082in

\tableofcontents

\pagebreak

\BLOCK{ if params or codelink or outputs_pickle or input_trace_pickles }
\section{Reproducing}
\BLOCK{ endif }

\begin{flushleft}
\BLOCK{ if params }
\subsection{Parameters}
%\verb=\VAR{key}= = \verb=\VAR{value}=
\begin{verbatim}
\BLOCK{ for key, value in params.items() }
\VAR{key} = \VAR{value}
\BLOCK{ endfor }
\end{verbatim}
\BLOCK{ endif }

\BLOCK{ if codelink }
\subsection{Code}
The code for this analysis can be found at:
\linebreak
\url{\VAR{codelink}}
\linebreak
%# Since the a block with "else" in it did not seem to be working.
The version of the code that generated this report \BLOCK{ if uncommitted_changes } had some changes not yet in version control. \BLOCK{ endif } \BLOCK{ if not uncommitted_changes } is the exact version of the code behind the link.
\BLOCK{ endif }

\BLOCK{ endif }

\BLOCK{ if outputs_pickle }
\subsection{Input}
Started analysis from computed values in \verb=\VAR{outputs_pickle}=.

The values in this intermediate file were previously computed from extracted
fluorescence traces.
\BLOCK{ endif }

\BLOCK{ if input_trace_pickles }
\subsection{Input}
Used extracted fluorescence traces in these files:
\begin{verbatim}
\BLOCK{ for infile in input_trace_pickles }
\VAR{infile}
\BLOCK{ endfor }
\end{verbatim}
\BLOCK{ endif }

\BLOCK{ if fly_id2key_str }
\subsection{Fly ID key}
% TODO maybe convert this to a (potentially multi-page) table
\begin{itemize}
\BLOCK{ for fly_id, key_str in fly_id2key_str.items() }
  \item \VAR{fly_id}: \VAR{key_str}
\BLOCK{ endfor }
\end{itemize}
\BLOCK{ endif }

\BLOCK{ if params or codelink or outputs_pickle or input_trace_pickles }
\pagebreak
\BLOCK{ endif }

\end{flushleft}

%# \BLOCK{ if params or codelink or outputs_pickle or input_trace_pickles or }
%# \pagebreak
%# \BLOCK{ endif }

% TODO maybe put this stuff in a header or something? or at least center?
% maybe back off vertical margin a bit?
\section{Across-fly analyses}

\BLOCK{ for section in sections }
\subsection{\VAR{section[0]}}

\figureSeriesHere{\VAR{section[0]}}

\BLOCK{ for figpdf in section[1] }
\figureSeriesRow{
\figureSeriesElement{}{\includegraphics[width=\textwidth,keepaspectratio]{\VAR{figpdf}}}
}
\BLOCK{ endfor }
\pagebreak

\BLOCK{ endfor }


\section{Within-fly analyses}

% TODO some way to reduce vspace between rows a bit? couldn't tell from
% figureSeries docs...
% w/ default vspace, 0.25\textheight seems about most I can do to fit 3 rows
% (0.3 was too much)

% More than ~0.49 times \textwidth seems to breakup the rows.
% Goal is to get 2 figures included per row, to have control and kiwi
% experiments (within each fly) side-by-side.

\BLOCK{ for section in paired_sections }
\subsection{\VAR{section[0]}}

\figureSeriesHere{\VAR{section[0]}}

% TODO pass in batch num + figure out textwidth factor from that
% 0.49 worked w/o creating new lines for batch(2).

% Note: in final latex, any blank lines between figureSeriesElements
% seems to make things take their own rows (for some unclear reason),
% so it's important that there are no blank lines below, around the
% BLOCK directives. (Previously I had empty lines here to visually
% separate the two if statements, and that seemed sufficient to cause
% failure of grouping figures into rows.)
\BLOCK{ for row_figpdfs in section[1]|batch(3) }
\figureSeriesRow{
\BLOCK{ for figpdf in row_figpdfs }
\BLOCK{ if figpdf }
\figureSeriesElement{}{\includegraphics[width=0.32\textwidth,height=0.25\textheight,keepaspectratio]{\VAR{figpdf}}}
\BLOCK{ endif }
\BLOCK{ if not figpdf }
\figureSeriesElement{}{\includegraphics[width=0.32\textwidth]{empty_placeholder.pdf}}
\BLOCK{ endif }
\BLOCK{ endfor }
}
\BLOCK{ endfor }

%# Was previously in a single figure definition in loop over figpdfs:
%# % https://tex.stackexchange.com/questions/8810
%# \BLOCK{ if filename_captions }
%# \cprotect\caption{\verb=\VAR{figpdf}=}
%# \BLOCK{ endif }
%# See: https://tex.stackexchange.com/questions/122786
%# if trying to adapt to subfloat case.

\pagebreak

\BLOCK{ endfor }


\end{document}
