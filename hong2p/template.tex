
\documentclass{article}
\usepackage{verbatim}

% https://tex.stackexchange.com/questions/6073
% https://tex.stackexchange.com/questions/195521
% I'm just using this package to add the "max width" optional key to
% includegraphics. Otherwise, it just wraps graphicx as far as I'm
% concerned.
\usepackage[export]{adjustbox}

\usepackage{fancyhdr}

\usepackage{figureSeries}

\usepackage{geometry}
% Comment after finding appropriate frame boundaries.
%\usepackage{showframe}
% From: https://tex.stackexchange.com/questions/39383/determine-text-width
% Required for printing \textwidth, etc, w/ particular units below.
%\usepackage{layouts}

% TODO what was this for?
\newgeometry{vmargin={15mm}, hmargin={12mm,17mm}}

% not sure this is working as intended... section seems displayed twice, once again
% before first figure in section
%
% If you don't want the "Figure: " prefix
\captionsetup[figure]{labelformat=empty}

% TODO what was this for?
\captionsetup[sub]{labelformat=empty}

% Some stack overflow comment said that although the \verb functionality is in LaTeX by
% default, the implementations in this package are "better", and override the defaults.
\usepackage{verbatimbox}

\usepackage{hyperref}
\hypersetup{
    colorlinks=true,
    linkcolor=blue,
    filecolor=magenta,
    urlcolor=cyan,
}
\urlstyle{same}

% TODO was it actually important they were PDFs (as opposed to e.g. pngs)?
% TODO can plots be in any subdirectory? what if i have multiple directories i wanna
% pull from?
\graphicspath{{\VAR{fig_dir}/}}

% For headers
\pagestyle{fancy}

% Clear existing header/footer entries
\fancyhf{}


\begin{document}

\BLOCK{ if header }
\begin{verbbox}\VAR{header}\end{verbbox}
\fancyhead[C]{\theverbbox}
\BLOCK{ endif }

% TODO try to replace section[0/1] w/ named variables, if jinja syntax supports
\BLOCK{ for section in sections }
% TODO what is stored in section[0] and section[1]? (section name and list of plots i
% think)
% TODO replace w/ just \section (and remove section above?)
%%\subsection{\VAR{section[0]}}
%\section{\VAR{section[0]}}

\figureSeriesHere{}
% TODO maybe just make the section names take up less space (/ add to a variable part of
% header maybe?). make display conditional on some flag in python fn compiling this?
%\figureSeriesHere{\VAR{section[0]}}

\BLOCK{ for fig_path in section[1] }
\BLOCK{ if filename_captions }
\begin{verbbox}\VAR{fig_path}\end{verbbox}
\BLOCK{ endif }
\figureSeriesRow{
\figureSeriesElement{\BLOCK{ if filename_captions }\theverbbox\BLOCK{ endif }}{\includegraphics[max width=\textwidth,keepaspectratio]{\VAR{fig_path}}}
}

%# TODO is this a valid test in this context?
%\BLOCK{ if fig_path in pagebreak_after }
%\pagebreak
%\BLOCK{ endif }

\BLOCK{ endfor }
\pagebreak

\BLOCK{ endfor }

%\section{Within-fly analyses}
%
%% TODO some way to reduce vspace between rows a bit? couldn't tell from
%% figureSeries docs...
%% w/ default vspace, 0.25\textheight seems about most I can do to fit 3 rows
%% (0.3 was too much)
%
%% More than ~0.49 times \textwidth seems to breakup the rows.
%% Goal is to get 2 figures included per row, to have control and kiwi
%% experiments (within each fly) side-by-side.
%
%\BLOCK{ for section in paired_sections }
%\subsection{\VAR{section[0]}}
%
%\figureSeriesHere{\VAR{section[0]}}
%
%% TODO pass in batch num + figure out textwidth factor from that
%% 0.49 worked w/o creating new lines for batch(2).
%
%% Note: in final latex, any blank lines between figureSeriesElements
%% seems to make things take their own rows (for some unclear reason),
%% so it's important that there are no blank lines below, around the
%% BLOCK directives. (Previously I had empty lines here to visually
%% separate the two if statements, and that seemed sufficient to cause
%% failure of grouping figures into rows.)
%\BLOCK{ for row_fig_paths in section[1]|batch(3) }
%
%\BLOCK{ if filename_captions }
%\BLOCK{ for fig_path in row_fig_paths }
%\BLOCK{ if fig_path }
%%# Converting from int loop counter to letter because \v1 and \v2 led to
%%# seemingly incorrect PDF output in a test case, but \va and \vb worked.
%%# I also tried this, but chr is undefined within the templater context.
%%# \VAR{ chr(97 + loop.index0) }
%%# TODO possible to factor stuff to get char into a fn somehow, so
%%# i don't need to redefine it three times?
%\begin{myverbbox}[\tiny]{\v\VAR{'abcdefghijklmnopqrstuvwxyz'[loop.index0]}}\VAR{fig_path}\end{myverbbox}
%\BLOCK{ endif }
%
%\BLOCK{ if not fig_path }
%\begin{myverbbox}{\v\VAR{'abcdefghijklmnopqrstuvwxyz'[loop.index0]}}\end{myverbbox}
%\BLOCK{ endif }
%\BLOCK{ endfor }
%
%\BLOCK{ endif }
%\figureSeriesRow{
%\BLOCK{ for fig_path in row_fig_paths }
%\BLOCK{ if fig_path }
%\figureSeriesElement{\BLOCK{ if filename_captions }\v\VAR{'abcdefghijklmnopqrstuvwxyz'[loop.index0]}\BLOCK{ endif }}{\includegraphics[width=0.32\textwidth,height=0.25\textheight,keepaspectratio]{\VAR{fig_path}}}
%\BLOCK{ endif }
%\BLOCK{ if not fig_path }
%\figureSeriesElement{}{\includegraphics[width=0.32\textwidth]{empty_placeholder.pdf}}
%\BLOCK{ endif }
%\BLOCK{ endfor }
%}
%
%%# TODO uncomment and get working after getting similar section above working
%%# TODO is fig_path still defined here (from loop above)
%%# if not, how to accomplish this?
%%#\BLOCK{ if fig_path in pagebreak_after }
%%#\pagebreak
%%#\BLOCK{ endif }
%
%\BLOCK{ endfor }
%
%\pagebreak
%
%\BLOCK{ endfor }
%

\end{document}
